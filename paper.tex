%-----------------------------------------------------------------------------
%
%               Template for sigplanconf LaTeX Class
%
% Name:         sigplanconf-template.tex
%
% Purpose:      A template for sigplanconf.cls, which is a LaTeX 2e class
%               file for SIGPLAN conference proceedings.
%
% Guide:        Refer to "Author's Guide to the ACM SIGPLAN Class,"
%               sigplanconf-guide.pdf
%
% Author:       Paul C. Anagnostopoulos
%               Windfall Software
%               978 371-2316
%               paul@windfall.com
%
% Created:      15 February 2005
%
%-----------------------------------------------------------------------------


\documentclass[preprint]{sigplanconf}

% The following \documentclass options may be useful:
%
% 10pt          To set in 10-point type instead of 9-point.
% 11pt          To set in 11-point type instead of 9-point.
% authoryear    To obtain author/year citation style instead of numeric.

%\usepackage{amsmath}
\usepackage{listings}
\usepackage{color}
\usepackage[style=alphabetic, sorting=nyt]{biblatex}
\addbibresource{bibliography.bib}

\lstset{
  language=Lisp,                % choose the language of the code
  columns=fixed,basewidth=.5em,
  basicstyle=\small\ttfamily,       % the size of the fonts that are used for the code
  %numbers=left,                   % where to put the line-numbers
  %numberstyle=\small\ttfamily,      % the size of the fonts that are used for the line-numbers
  %stepnumber=1,                   % the step between two line-numbers. If it is 1 each line will be numbered
  %numbersep=5pt,                  % how far the line-numbers are from the code
  %backgroundcolor=\color{white},  % choose the background color. You must add \usepackage{color}
  %showspaces=false,               % show spaces adding particular underscores
  showstringspaces=false,         % underline spaces within strings
  %showtabs=false,                 % show tabs within strings adding particular underscores
  frame=single,           % adds a frame around the code
  %tabsize=2,          % sets default tabsize to 2 spaces
  captionpos=t,           % sets the caption-position to bottom
  breaklines=true,        % sets automatic line breaking
  breakatwhitespace=true,    % sets if automatic breaks should only happen at whitespace
  %escapeinside={\%*}{*)},          % if you want to add a comment within your code
}

\begin{document}

\conferenceinfo{WXYZ '05}{date, City.} 
\copyrightyear{2014}
\copyrightdata{[to be supplied]} 

%\titlebanner{banner above paper title}        % These are ignored unless
\preprintfooter{short description of paper}   % 'preprint' option specified.

\title{Title}
%\subtitle{Subtitle Text, if any}

\authorinfo{Author1 Name}
           {Company}
           {Email@gmail.com}

\authorinfo{Author2 Name}
           {Company}
           {Email@gmail.com}

\maketitle

% Molehills, not mountains! - SPJ
% ie. don't try and solve everything ambitiously,
% attack a very specific problem with specific examples

% 1st page
%
% 1. Intro
% - write *explicitly* "the main idea of this paper is"
%   - but only once you have set up enough scaffolding for context
% 2. Contributions
% - write early, and rewrite
%   - need to know what the contributions are early
% - list of contributions drive everything in the paper
% - consider bullet points
%   - make it *so* easy to read
%   - *don't* lose readers here
% - *every* claim has a forward reference to *evidence* that supports the claim
% - pg 1 = spec, rest of paper = impl
% - Contributions *must* be *refutable*
%    - NOT We describe/study the properties of this type system
%    - YES We prove the soundness of ...
%    - *crunchy!* refutable contributions
%      - "think celery, not soggy pasta claims!"
% 
% Give citations *during the paper* (or, we will discuss this related work in section ..)
% "You know all about the literature! You're just getting there at the end.. (of the paper)"

% - Related work
%   - don't make other people's work look bad!
%   - you have unlimited love to spread around
%   - praise inspiring papers!
%     - credit to the author
%     - it won't make your work looks worse
%   - "in this dimension, our work is better (worse!!)"
%     - point out your shortcomings before your reviews
%     - not just honest/scholarly, just good tactics

% Put your readers first
% - "the middle section"
%   - easier to write (the stuff in your head)
% - don't present the general solution first
%   - "Consider a semilattice B over hyperstructural interposed structure C ...."
%   - might be obvious to you (you've learnt this!) but not *motivating* for readers
%   - what would you do at the whiteboard?
%     - explain the intuition *first*
%     - even if the reader leaves the paper here, they still still take something away
% - Conveying the intuition
%   - examples! (at the whiteboard..)
%   - we can't help giving examples in conversation, but in papers we can't help giving the general case first!!

% Don't bring your readers down blind alleys
% - research is soaked in (your) blood
%   - don't bring your readers down the dead ends
% - *only* explore bind alleys that are *so* obvious the reader is itching to correct you as you explain your
%   solution!

\begin{abstract}
  % https://plg.uwaterloo.ca/~migod/research/beckOOPSLA.html
\end{abstract}

%\category{CR-number}{subcategory}{third-level}

% - what is the problem?
% - why is it interesting, challenging, important?
% - what have others done before?
% - what did you do?
% - how did you evaluate it?
% - what contribution have you made?
% - how would you continue this in the future?

%\terms
%term1, term2

%\keywords
%keyword1, keyword2

\section*{Introduction}

% Start immediately with an example, like at a whiteboard

Here's how to make an inline code sample: \lstinline|lift-chan|.

\begin{lstlisting}
;; A normal code listing.
(inc 1)
\end{lstlisting}

This is how to cite~\cite{Tob10} something.

% Contributions

Here are our main contributions.

\begin{itemize}
  \item We discuss (Section~\ref{first})
  \item We demonstrate (Section~\ref{second})
\end{itemize}

\section{First section}
\label{first}

\section{Second section}
\label{second}

\begin{lstlisting}[float]
;; How to make a floating listing
(+ 1 2)
\end{lstlisting}

\printbibliography[title=References]

\end{document}
